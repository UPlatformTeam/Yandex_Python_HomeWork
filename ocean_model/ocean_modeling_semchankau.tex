\documentclass{article}

\usepackage[english, russian]{babel}
\usepackage[utf8x]{inputenc}
\usepackage{amsmath}
\usepackage{amsfonts}
\usepackage{amssymb}
\usepackage{amsthm} 
\usepackage{cite}  


\begin{document}

\title {Отчет о цикличности}
\author {Семченков Алексей\\Московский Физико-Технический Институт}
\maketitle

\section{Модель}

Наша модель представляет собою таблицу высотой $H$ и шириной $W$. Каждая клетка или пуста ('$0$'), или занята акулой ('$S$'), или занята обычной рыбой ('$F$'). Для удобства реализации модель окаймлена пустыми клетками, в которые нельзя заплывать. \\\\
В каждый момент времени каждая рыба, если это возможно, передвигается в одну из соседних клеток случайным образом. Если же это акула, то она предпочитает плыть в те клетки, где находится обычная рыба, чтобы ее съесть.\\\\
Классы Shark и Fish обладают такими атрибутами, как \textit{division\_period} и \textit{division\_counter}. В каждый момент времени \textit{division\_counter} каждого существа увеличивается на единицу. Как только \textit{division\_counter} перевалит за \textit{division\_period}, существо размножается (в соседнюю клетку помещается существо того же вида), а \textit{division\_counter} обнуляется.\\\\
У класса Shark есть также свой собственные атрибуты --- \textit{life\_durability} и \textit{life\_counter}. Каждый момент времени \textit{life\_counter} увеличивается на единичку. Как только он переваливает через значение \textit{life\_durability}, акула умирает. Если же она съедает существа класса Fish, то ее \textit{life\_counter} обнуляется, и акула может прожить еще какое-то время в беспечности. \\\\
В нашей модели размеры таблицы равны $50\times 50$, изначально приблизительно $20\%$ клетов заняты акулами, а $60\%$ заняты обычной рыбой. Длительность жизни каждый акулы выбирается произвольного из промежутка $[5,7]$, а период размножения акулы равен удвоенной продолжительности жизни. Период размножения каждый обычной рыбы выбирается произвольно из промежутка $[2, 5]$.

\section{Полученный результат}

В приложенном png-файле по оси $x$ отмечено количество итераций, на котором изучалась модель (всего $2000$ итераций), по оси $y$ отмечены соответственные количества обычных рыбок (красные точки) и акул (синие точки).\\\\ 
Из графика видно, что периодичность системы составляет приблизительно $60-70$ итераций, после которых система возвращается в состояние максимума по числу обычных рыбок. Как видим, всего есть четыре основных состояния системы:\\
1) Количество обычных рыбок находится на максимуме, где-то $60-75\%$, акул при этом около $7\%$ от размера таблицы;\\
2) Акулы достигают своего максимума в $10-11\%$ от мирового населения, обычных рыбок к этому моменту становится около всего $30-35\%$\\
3) Популяция обычных рыбок достигает своего минимума --- $18-22\%$ от количества клеток в таблице, акул в это время опять около $7\%$\\
4) Акулы достигают своего минимума в $5\%$, обычные рыбки возвращаются к своему среднему в $30-35\%$

\end{document}